\section{INTRODUCCIÓN}
\subsection{FUNDAMENTACIÓN}
La síntesis del habla consiste en la tarea de generar una voz humana a partir de otro tipo de entrada, ya sea texto, movimiento de labios o fonemas. En la mayoría de sus aplicaciones modernas, estos sistemas toman el texto como método de entrada. Esto se debe en parte a los avances en el campo del procesamiento del lenguaje natural. Un sistema de texto a voz (TTS por sus siglas en inglés, text to speech system) apunta a convertir el lenguaje escrito en discurso humano audible.

Históricamente esta tarea fue llevada a cabo por sistemas que concatenan fonemas pre-grabados (sistemas concatenativos) o que modelan un audio a través de parámetros acústicos definidos arbitrariamente (sistemas paramétricos). A lo largo de la última década, hubo avances en el poder computacional que permitieron explorar y desarrollar diversos modelos de TTS basados en el aprendizaje automático profundo (\textit{Deep Learning}), a partir de diversas metodologías: En 2016 el equipo de DeepMind introduciría WaveNet \cite{wavenet} revolucionando el campo del TTS con el primer modelo que sintetizaba el habla humana muestra por muestra. Este pilar fue seguido por numerosos avances y mejoras basadas en sistemas de paralelización \cite{paralel}, transformadores \cite{fastspeech} y sistemas de tipo Flow \cite{waveglow}.

Evaluar la calidad de estas distintas soluciones implica, entre otras cosas, juzgar la “naturalidad” de la voz humana generada. El estándar para realizar esa evaluación son las pruebas subjetivas, realizadas sobre sistemas entrenados con bases de datos estandarizadas, usualmente en idiomas inglés o chino. El Mean Opinion Score (MOS) \cite{itu} (puntaje promedio subjetivo) es el método más frecuentemente utilizado para llevar a cabo esa prueba. Dicha métrica tiene un rango de 1 a 5, en la que el habla humana real yace entre las puntuaciones de 4,5 a 4,8. El test MOS se conduce sobre las voces sintetizadas para dar un idea de que tan naturales son los resultados de los sistemas TTS.

Realizar un test subjetivo es costoso monetaria y temporalmente, e indefectiblemente presenta una barrera a la hora de evaluar pequeñas modificaciones o iteraciones en el desarrollo de un sistema TTS. Este documento detalla el desarrollo de un procedimiento de evaluación objetiva para sistemas de texto a voz. Dicha evaluación busca tener un alto grado de correlación con los resultados de las pruebas subjetivas. Se planea ofrecer la métrica desarrollada de forma gratuita y como código abierto.

Intercambios Transorgánicos (Dir. Gala González Barrios) es un programa de investigaciones radicado en el Muntref Centro de Arte y Ciencia, IIAC, UNTREF. Desde este programa se realizan proyectos de investigación que desarrollan interfaces interactivas desde las artes electrónicas y las ingenierías en relación con el campo de la salud. En este momento se encuentran desarrollando un sistema TTS en español argentino, orientado a funcionar como parte de una prótesis para personas que se encuentren en la situación de comprometer su voz, parcial o totalmente. La investigación planteada en esta tesis busca proveer una herramienta para evaluar y ayudar al progreso y desarrollo de dicha herramienta.

El trabajo propuesto es una investigación cuantitativa de alcance exploratorio. Su propósito es el de brindar a la comunidad de investigadores que desarrollan sistemas de texto a voz, una evaluación objetiva automatizada que presente un alto grado de correlación con las pruebas subjetivas que conforman el estándar de la industria para juzgar el habla. Se plantea extraer un descriptor de cada audio a juzgar, y entrenar una pequeña red neuronal de forma supervisada, de modo que la misma pueda predecir el valor MOS que obtendría el audio si fuese juzgado subjetivamente por un grupo de individuos.

\subsection{OBJETIVOS}
\subsubsection{OBJETIVO GENERAL}
El diseño, implementación y validación de un sistema computacional capaz de predecir la preferencia subjetiva promedio (MOS), sobre distintas voces sintetizadas por computadoras. La investigación se condujo en el idioma castellano.


\subsubsection{OBJETIVOS ESPECÍFICOS}
Se proponen los siguientes objetivos específicos:
\begin{itemize}
    
 \item \textbf{Recolección de audios sintetizados.} 
Recolectar audios sintetizados por sistemas TTS de variada calidad de clonado de voz y procedencia, además de audios de hablantes humanos reales. 
\item \textbf{Transformación de audios recolectados.}
Alterar la calidad de una porción de los audios recolectados mediante distintas técnicas de procesado de señales y voz, obteniendo así una base de datos mas balanceada.
\item \textbf{Extracción de descriptores objetivos de cada audio.}
Extraer representaciones vectorizadas de cada audio mediante una red neuronal convolucional que aprenda a representar la características acústicas de cada voz a evaluar. Estos embeddings son utilizados para reconocimiento de hablantes se extraen mediante una red neuronal que deberá ser configurada y posiblemente re-entrenada para funcionar con el idioma castellano.
\item \textbf{Diseño de una prueba subjetiva para etiquetar los audios. }
Diseñar y llevar a cabo una prueba subjetivas para obtener una puntuación para cada audio obtenido, seguido de una validación de los datos obtenidos.
\item \textbf{ Diseño de una red neuronal para predecir la naturalidad de cada audio.}
Entrenar una pequeña red neuronal de forma supervisada, con los audios recolectados como entrada y sus calificaciones MOS como salida deseada.La función de costo y el ajuste de la red tendrán como objetivo acercar sus predicciones a los valores correctos MOS recolectados. Para poner a prueba el modelo entrenado, se reserva una parte del conjunto de datos recolectados para llevar a cabo una evaluación del sistema.

\end{itemize}

\subsection{ESTRUCTURA DE LA INVESTIGACIÓN}

En capítulo 2 se detalla un marco teórico vinculado a los procesos detrás de las distintas implementaciones posibles para sintetizar voces artificialmente, la evaluación subjetiva MOS, y la predicción de parámetros subjetivos mediante métricas objetivas. También se provee una breve explicación de las distintas técnicas detrás de los métodos de alteración de hablantes que se utilizaron en el transcurso de la investigación, así como también información vinculada la la vectorización de hablantes utilizada. En el capitulo 3 se presenta el estado del arte vinculado a la evaluación objetiva de sistemas TTS. El capítulo 4 consta del desarrollo de la investigación, en el cual se evidencian las distintas características de la base de datos obtenida, el diseño de la prueba subjetiva y el diseño e implementación de la red neuronal clasificadora de TTS. En el capitulo 5 y 6 se presentan y analizan los resultados obtenidos. Finalmente, en el capitulo 7 se informan las conclusiones de la investigación desarrollada, y el capitulo 8 ofrece posibles lineas de investigación futuras que se desprenden de los resultados obtenidos.

\newpage